\begin{sloppypar} % помогает в кириллическом документе выровнять текст по краям
\newpage % Так добавляется  новая страница
\section{РАЗРАБОТКА СТРУКТУРНОЙ СХЕМЫ УСТРОЙСТВА} %Объявили начало раздела

% Многие эксперты и исследователи пытались разработать неинвазивные глюкометры. Однако в настоящее время неинвазивное определение уровня глюкозы в крови все еще имеет проблемы, такие как чувствительность и сигналы фонового шума, которые необходимо преодолеть, а использование защитных покрытий, таких как pHEMA \cite{Ph}, может обеспечить минимально инвазивное и точное измерение изменений уровня глюкозы в крови.

% Таким образом, основной целью данного исследования является достижение точного и эффективного измерения уровня глюкозы в крови у пациентов с диабетом, а также уменьшение боли и дискомфорта во время процесса, что может улучшить качество жизни. Мы разрабатываем электрохимический датчик на основе массива микроигл, модуль схемы обнаружения глюкозы и модуль передачи для размещения в носимом устройстве, которое может непрерывно определять концентрацию глюкозы в интерстициальной жидкости [ISF] с низкой инвазивностью и передавать данные на мобильный телефон по Bluetooth, с точным измерением изменений концентрации глюкозы. Минимально инвазивная рана после использования показана на \ref{ris:Figures/struct2.png}. 
% \imghh{150mm}{Figures/struct2.png}{Минимально инвазивная рана после использования CGMS, эта небольшая рана практически не вызывает боли и кровотечения}


% Концепция микропереноса была применена для переноса глюкозооксидазы на датчик матрицы микроигл. Этот набор микроигл был приобретен у RichHealth Technology. Датчик массива микроигл показан на рисунке  \ref{ris:Figures/struct1.png}., который включает в себя три части: рабочий электрод (WE), противоэлектрод (CE) и электрод сравнения (RE). Массив микроигл изготовлен из нержавеющей стали SUS316L с использованием процесса штамповки металла. Каждая микроигла имеет длину 1 мм, ширину 0,25 мм и толщину 0,1 мм. Затем противоэлектрод покрывают золотом, рабочий электрод покрывают золотом и ПАНИ, а электрод сравнения покрывают Ag/AgCl. Каждый массив WE имеет площадь 3 мм×3 мм и состоит из 3×4 микроиглы. Рабочая площадь каждого массива микроигл составляет около 1,2 мм.2. Все иглы скошены для предотвращения нестабильности сигнала из-за отслоения глюкозооксидазы при прокалывании под кожей


% \imghh{150mm}{Figures/struct1.png}{Структура датчика матрицы микроигл;}


Структурная схема устройства представлена на рисунке \ref{ris:Figures/struct.png}.
\imghh{150mm}{Figures/struct.png}{Структурная схема устройства}


Принцип работы устройства заключается в следующем: микроконтроллер управляет подсветкой и камерой. С помощью камеры получаем изображение кожи, которое будет обработано посредством нейросети внутри микроконтроллера. Питается вся система с помощью встроенного аккумулятора, также предусмотрено питание/заряд аккумулятора от USB, с помощью внешнего блока питания или от порта ПК. 

Результаты анализа выводятся на ЖК-экран. Снимки кожи и результаты их обработки передаются по внешнему модулю Bluetooth, подключенному к микроконтроллеру, и могут быть записаны во Flash-накопитель или SD-карту для дальнейших исследований.


\end{sloppypar}
