\begin{sloppypar} % помогает в кириллическом документе выровнять текст по краям
\newpage % Так добавляется  новая страница
\section{РАЗРАБОТКА ПРИНЦИПИАЛЬНОЙ СХЕМЫ УСТРОЙСТВА} %Объявили начало раздела


\subsection{Выбор усилителя биопотенциалов}
Схема подает напряжение, чтобы вызвать окислительно-восстановительную реакцию на рабочем электроде. Этот заряд усиливает токовую реакцию электрода, и, наконец, чувствительная схема усиливает сигнал электрода. Противоэлектрод (CE) находится под отрицательным напряжением относительно рабочего электрода. Кроме того, ток, генерируемый на выводе рабочего электрода, составляет менее 100 нА при концентрации мг/дЛ. Поскольку этого тока недостаточно, для преобразования тока в выходное напряжение требуется трансимпендансный усилитель на операционном усилителе с чрезвычайно низким током смещения. Его схема показана на  рисунке \ref{ris:Figures/struct-tr.png}. 
\imghh{80mm}{Figures/struct-tr.png}{Трансимпедансный усилитель}



С этой целью мы использовали операционный усилитель MAX9913 \cite{MAX9913}, который подходит для аналогичных применений. При комнатной температуре имеет низкий ток смещения и хорошие характеристики защиты от шумовых помех. Типичная схема работы инструментального усилителя из даташита MAX9913 приведена на рисунке  \ref{ris:Figures/max.png}. 
\imghh{80mm}{Figures/max.png}{Cхема включения MAX9913}

Нумерация и назначение выводов MAX9913 приведено ниже (рисунки \ref{ris:Figures/max1.png}, \ref{ris:Figures/max2.png}).
\imghh{80mm}{Figures/max1.png}{Распиновка MAX9913}

\imghh{150mm}{Figures/max2.png}{Назначение выводов MAX9913}




\subsection{Выбор микроконтроллера}
С учетом технического задания микроконтроллер должен обладать следующими свойствами:
\begin{onehalfspace}
	\begin{itemize}
		\item[--]Интерфейс для работы с термодатчиком :  $I^2$C;
		\item[--]Интерфейс для работы с внешней флеш-памятью: SPI или $I^2$C;
		\item[--]Для передачи данных по Bluetooth: встроенный стек протокола Bluetooth;
		\item[--]Малое энергопотребление;
	\end{itemize}
\end{onehalfspace}

Для решения задачи был выбран микроконтроллер STM32WB55RCV6 фирмы ST Microelectronics \cite {STM}.STM32WB55 содержит два производительных ядра ARM-Cortex:
\begin{onehalfspace}
	\begin{itemize}
		\item[--] ядро ARM® -Cortex® M4 (прикладное), работающее на частотах до 64 МГц, для пользовательских задач имеется модуль управления памятью, модуль плавающей точки, инструкции ЦОС (цифровой обработки сигналов), графический ускоритель (ART accelerator);
		\item[--] ядро ARM®-Cortex® M0+ (радиоконтроллер) с тактовой частотой 32 МГц, управляющее радиотрактом и реализующее низкоуровневые функции сетевых протоколов;
	\end{itemize}
\end{onehalfspace}


Основные характеристики:
\begin{onehalfspace}
	\begin{itemize}
		\item[--] типовое энергопотребление 50 мкА/МГц (при напряжении питания 3 В);
		\item[--] потребление в режиме останова 1,8 мкА (радиочасть в режиме ожидания (standby));
		\item[--] потребление в выключенном состоянии (Shutdown) менее 50 нА;
		\item[--] диапазон допустимых напряжений питания 1,7…3,6 В (встроенный DC-DC–преобразователь и LDO-стабилизатор);
		\item[--] рабочий температурный диапазон -40…105°С.
	\end{itemize}
\end{onehalfspace}


Структурная схема микроконтроллера приведена на рисунке \ref{ris:Figures/stm32.png}, а назначение выводов портов корпуса на рисунке \ref{ris:Figures/stm32io.png}.

\imghh{90mm}{Figures/stm32io.png}{Назначение выводов}


\imghh{110mm}{Figures/stm32.png}{Структурная схема}


Подключение будет осуществляться согласно типовой схеме из Application note\cite {STM_an}(рисунок \ref{ris:Figures/stm32an.png}).
\imghh{160mm}{Figures/stm32an.png}{Типовая схема подключения STM32WB55}






\begin{sloppypar} % помогает в кириллическом документе выровнять текст по краям


\subsection{Выбор встроенного носителя}
 В качестве носителя информации выберем последовательную FLASH-память серии W25Q \cite{W25Q}. Данная последовательная память может быть различной ёмкости — 8, 16, 32, 64, 128, 256 Мбит и т. д. Подключается такая память по интерфейсу SPI, а также по многопроводным интерфейсам Dual SPI, Quad SPI и QPI. Мы подключим данную микросхему по обычному интерфейсу SPI.
Необходимый объем памяти для наших нужд рассчитывается по формуле: объем памяти= 16*1*60*24 = 23040bit \approx 23 Mbit
где 16 - необходимый объем памяти для хранения одного измерения в битах, 1- число измерений в минуту, 60 - количество минут в одном часе, 24 - количество часов в сутках.
Таким образом, для хранения данных в течение 24 часов нам нужно выбрать FLASH-память с ёмкостью 32 Mbit.

 
 
 

Краткие основные характеристики W25Q:

\begin{onehalfspace}
	\begin{itemize}
		\item[--]Потребляемая мощность и температурный диапазон:
		\item[--]Напряжение питания 2.7…3.6 В
		\item[--]Типичный потребляемый ток: 4 мА (активный режим), <1 мкА (в режиме снижения мощности)
		\item[--]Рабочий температурный диапазон -40°C…+85°C.
	\end{itemize}
\end{onehalfspace}

Гибкая архитектура с секторами размером 4 кбайт:
\begin{onehalfspace}
	\begin{itemize}
		\item[--]Посекторное стирание (размер каждого сектора 4 кбайт)
		\item[--]Программирование от 1 до 256 байт
		\item[--]До 100 тыс. циклов стирания/записи
		\item[--] 20-летнее хранение данных
	\end{itemize}
\end{onehalfspace}


Максимальная частота работы микросхемы:
\begin{onehalfspace}
	\begin{itemize}
		\item[--]104 МГц в режиме SPI
		\item[--]208/416 МГц — Dual / Quad SPI
	\end{itemize}
\end{onehalfspace}

Также микросхема существует в различных корпусах, но в большинстве случаев распространён корпус SMD SO8. Распиновка микросхемы следующая(рисунок \ref{ris:Figures/w25io.png}).
\imghh{90mm}{Figures/w25io.png}{Распиновка W25Q32}

Описание выводов из  \cite {W25Q}(рисунок \ref{ris:Figures/w25qan.png}).
\imghh{160mm}{Figures/w25qan.png}{Описание выводов W25Q32}
К микроконтроллеру подключается по стандартному интерфейсу SPI.



\end{sloppypar}
 



\begin{sloppypar} % помогает в кириллическом документе выровнять текст по краям


\subsection{Выбор встроенного носителя}
Для точного измерения температуры окружающей среды (в том числе места контакта устройства с телом исследуемого) возникла необходимость установки периферийного высокоточного датчика, который обеспечивал бы схему информацией о температуре и в случае перегрева предпринимались бы необходимые меры для исключения снятия неточных данных. Таковым был выбран датчик от фирмы Texas Instruments TMP102 \cite{TMP102}. Типовая схема включения и конфигурация выводов изображены на рисунках \ref{ris:Figures/tmpcnct.png}  и \ref{ris:Figures/tmp.png}. Информацию с него будем получать по шине $I^2$C

Также микросхема существует в различных корпусах, но в большинстве случаев распространён корпус SMD SO8.
\imghh{90mm}{Figures/tmpcnct.png}{Типичная схема включения TMP102}

\imghh{160mm}{Figures/tmp.png}{Описание выводов и распиновка TMP102}
К микроконтроллеру подключается по стандартному интерфейсу $I^2$C.





\end{sloppypar}
 


\subsection{Схема питания}
Питание данного устройства спроектировано от часовой батарейки с напряжением питания 1.5В. В связи с тем, что практически все элементы схемы требуют напряжение питания больше 3 вольт, используем повышающий DC-DC преобразователь напряжения. Возьмем модель от Texas Instruments TPS63802 2-A\cite{TPS63802}. Конфигурация выводов изображена на рисунке \ref{ris:Figures/TPS63802.png}. Данный конвертер поднимет напряжение с 1.8 вольт до 3.5. 


\imghh{160mm}{Figures/TPS63802.png}{Распиновка и описание выводов TPS63802}

В целях обеспечения сохранности первоначального вида сигнала, поставим LOW-DROP-OUT регулятор напряжения перед питанием усилителя биопотенциалов. 

Возьмем R1517S331D-E2-FE от компании RICOH[3]. На рисунке \ref{ris:Figures/r15.png} показано типичное включение данного регулятора в цепь. Данный регулятор обезопасит усилитель от дребезгов питания.


\imghh{120mm}{Figures/r15.png}{Типовая схема включения R1517S331D}










\end{sloppypar}
